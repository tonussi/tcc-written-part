% ----------------------------------------------------------
\chapter{Introdução}
% ----------------------------------------------------------

As orientações aqui apresentadas são baseadas em um conjunto de normas elaboradas pela \gls{ABNT}. Além das normas técnicas, a Biblioteca também elaborou uma série de tutoriais, guias, \textit{templates} os quais estão disponíveis em seu site, no endereço \url{http://portal.bu.ufsc.br/normalizacao/}.

Paralelamente ao uso deste \textit{template} recomenda-se que seja utilizado o \textbf{Tutorial de Trabalhos Acadêmicos} (disponível neste link \url{https://repositorio.ufsc.br/handle/123456789/180829}) e/ou que o discente \textbf{participe das capacitações oferecidas da Biblioteca Universitária da UFSC}.

Este \textit{template} está configurado apenas para a impressão utilizando o anverso das folhas, caso você queira imprimir usando a frente e o verso, acrescente a opção \textit{openright} e mude de \textit{oneside} para \textit{twoside} nas configurações da classe \textit{abntex2} no início do arquivo principal \textit{main.tex} \cite{abntex2classe}.

Conforme a \href{https://repositorio.ufsc.br/bitstream/handle/123456789/197121/RN46.2019.pdf?sequence=1&isAllowed=y}{Resolução NORMATIVA nº 46/2019/CPG} as dissertações e teses não serão mais entregues em formato impresso na Biblioteca Universitária. Consulte o Repositório Institucional da UFSC ou sua Secretaria de Pós Graduação sobre os procedimentos para a entrega. 

\nocite{NBR6023:2002}
\nocite{NBR6027:2012}
\nocite{NBR6028:2003}
\nocite{NBR10520:2002}

\section{Motivações}

% ----------------------------------------------------------
\section{Objetivos}
% ----------------------------------------------------------

Nas seções abaixo estão descritos o objetivo geral e os objetivos 
específicos.

% ----------------------------------------------------------
\subsection{Objetivo Geral}
% ----------------------------------------------------------

Descrição...

% ----------------------------------------------------------
\subsection{Objetivos Específicos}
% ----------------------------------------------------------

Descrição...
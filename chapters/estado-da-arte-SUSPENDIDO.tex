\chapter{Estado da Arte}

Nesse capítulo é feita uma pesquisa e revisão do estado mais atual da pesquisa referente ao uso de microsserviços e replicação por máquina de estados. Utilizando metodologia básica de pesquisa com termos chaves, a fim de encontrar resultados apurados para análise.

Para isso realizamos uma revisão simplificada, que consiste em pesquisa para: identificar, avaliar e interpretar uma quantidade específica e qualificada de trabalhos pertinentes ao questionamento dessa pesquisa.

\section{Contextualização da Pesquisa}

Após o estudo da fundamentação teórica, realizando uma pesquisa bibliográfica sobre os principais assuntos que envolvem esse trabalho que são: Gerência de Projeto, e Jogos Educacionais, foram também realizadas pesquisas por livros e artigos científicos sobre os assuntos já expostos, em bases científicas reconhecidas, a saber, \textit{Google Scholar}\footnote{\url{https://scholar.google.com.br/}}, ACM Digital Library \footnote{\url{https://dl.acm.org/}}, \textit{IEEEXplorer}\footnote{\url{https://ieeexplore.ieee.org/}}, \textit{Science Direct}\footnote{\url{https://www.sciencedirect.com/}}.

A seguir foram definidos as Perguntas de Pesquisa em Inglês e Português.

\begin{table}[!htb]
\caption{Criação e idealização da pesquisa}
\begin{center}
{
\footnotesize
\resizebox{\textwidth}{!}{
\begin{tabular}{
|p{7cm}|p{7cm}|
}
\hline
\cellcolor{gray!15}
\textbf{Perguntas de Pesquisa (Inglês)} &
\cellcolor{gray!15}
\textbf{Perguntas de Pesquisa (Português)} \\
\hline
What are the papers, until today, regarding microservices architecture and state machine replication for securing high disponibility for distributed systems, also known as fault-tolerance? & Quais foram as pesquisas, até hoje, relacionadas à arquitetura de microsserviços e replicação por máquina de estados, para assegurar alta disponibilidade em sistemas distribuídos, também conhecido como tolerância à falhas? \\
\hline
\end{tabular}
}
}
\end{center}
\label{tab:idealizacao}
\end{table}

A próxima tabela mostra à extração dos termos de pesquisa, que são levados em consideração para criar as chaves de busca em um padrão de linguagem descritiva de busca em base de dados científicas.

\begin{table}[!htb]
\caption{Extração dos termos relevantes para construir a busca}
\begin{center}
{
\footnotesize
\resizebox{\textwidth}{!}{
\begin{tabular}{
|p{7cm}|p{7cm}|
}
\hline
\cellcolor{gray!15}
\textbf{Termos de Pesquisa (Inglês)} &
\cellcolor{gray!15}
\textbf{Termos de Pesquisa (Português)} \\
\hline
"microservices", "state machine replication" &
"microsserviços", "replicação por máquina de estados" \\
\hline
\end{tabular}
}
}
\end{center}
\label{tab:extracao-termos}
\end{table}

Notar que microsserviços automaticamente remete à arquitetura. E replicação por máquina de estados remete à sistemas distribuídos, e garantia de disponibilidade, além disso remete também à tolerância à falhas \cite{schneider1990implementing}.

\subsection{Critérios de Inclusão e Exclusão}

Os termos de busca escolhidos podem resultar em artigos que não se encaixam dentro do contexto desse trabalho, como por exemplo, a decisão entre.

Assim, uma lista de critérios de inclusão e exclusão foi criada, determinando quais artigos podem ser potencialmente relevantes, e quais serão desconsiderados.

\textbf{Critérios de Inclusão}:

\begin{itemize}
\item Trabalhos que implementam uma solução de replicação por máquinas de estado em arquitetura de microsserviços.
\item Trabalhos que estão envolvidos com algoritmos que implementam consenso entre as partes, ou seja, os servidores de um sistema distribuído.
\item É desejável, mas não estritamente obrigatório, que os trabalhos usufruam de tecnologia para orquestração de contêineres (i.e. Kubernetes), além de uma tecnologia para contêineres (i.e. Docker).
\end{itemize}

\textbf{Critérios de Exclusão}:

\begin{itemize}
\item Trabalhos que gravitam em torno de: \textit{Blockchain}, Segurança de Redes de Computadores, e Cripto-moedas.
\item Trabalhos envolvendo Replicação Máquina de Estados Paralela, a versão em Paralela em específico.
\item Trabalhos envolvendo Inteligência Artificial, por exemplo: Uso de algoritmos: Genéticos, Redes Neurais, Detecção de Objetos, dentre outros.
\end{itemize}

\section{Critérios de Qualidade}

A seguir estão estabelecidos os critérios de qualidade. Esses critérios foram elaborados para filtrar os artigos científicos, para possibilitar a leitura e análise somente. Os critérios de qualidade foram então aplicados após a execução das strings de busca nas bases de conhecimento científicas.

\begin{itemize}
\item Os trabalhos devem ser objetivos claros, com informação relevante à sistemas distribuídos, arquitetura de microsserviços, replicação por máquinas de estados, e protocolo de consenso.
\item Os trabalhos devem explicitar claramente a relação de microsserviços com orquestração de contêineres.
\item Os trabalhos devem ter métodos de análise de desempenho específicos para Computação Distribuída.
\item Os trabalhos devem ter feito avaliação experimental.
\end{itemize}

\section{Dados a Serem Extraídos}

A seguir brevemente explicados alguns dados a serem extraídos dos trabalhos.

\begin{itemize}
\item Contextualização da pesquisa, tecnologias usada, algoritmos de consenso abordados, ferramentas de análise de sistemas distribuídos, citações relevantes.
\end{itemize}

\section{Execução da Busca e seus Resultados}

Na execução da busca por resultados, foram adquiridos os termos de busca  configurados conforme as tabelas a seguir. As tabelas a seguir apresentam os termos de pesquisa, a quantidade de artigos relevantes, e quantidade descartada devido aos critérios previamente mencionados.


\begin{table}[!htb]
\caption{Extração de resultados pela Science Direct}
\begin{center}
{
\footnotesize
\resizebox{\textwidth}{!}{
\begin{tabular}{
|p{0.8\linewidth}|p{0.2\linewidth}|
}
\hline
\cellcolor{gray!15}
\textbf{Busca} &
\cellcolor{gray!15}
\textbf{Quantidade} \\
\hline
{\tiny TITLE-ABSTR-KEY(“microservice”) AND
TITLE-ABSTR-KEY(“state machine replication”) AND NOT
TITLE-ABSTR-KEY(“blockchain”)} &
1 artigo relevante e 1 descartado \\
\hline
\end{tabular}
}
}
\end{center}
\label{tab:extracao-ScienceDirect}
\end{table}


\begin{table}[!htb]
\caption{Extração de resultados pela IEEEXplore}
\begin{center}
{
\footnotesize
\resizebox{\textwidth}{!}{
\begin{tabular}{
|p{0.8\linewidth}|p{0.2\linewidth}|
}
\hline
\cellcolor{gray!15}
\textbf{Busca} &
\cellcolor{gray!15}
\textbf{Quantidade} \\
\hline
{\tiny ("Abstract": “state machine replication” OR "Publication Title": “state machine replication” ) } &
34 artigos \\
\hline
\end{tabular}
}
}
\end{center}
\label{tab:extracao-IEEEXplore}
\end{table}


\begin{table}[!htb]
\caption{Extração de resultados pela Google Scholar}
\begin{center}
{
\footnotesize
\resizebox{\textwidth}{!}{
\begin{tabular}{
|p{0.8\linewidth}|p{0.2\linewidth}|
}
\hline
\cellcolor{gray!15}
\textbf{Busca} &
\cellcolor{gray!15}
\textbf{Quantidade} \\
\hline
{\tiny ("microservice" \space "state machine replication" \space -"blockchain" \space -"network security" \space -"artificial intelligence")} &
52 \\
\hline
\end{tabular}
}
}
\end{center}
\label{tab:extracao-GoogleScholar}
\end{table}

\begin{table}[!htb]
\caption{Extração de resultados pela \gls{ACM}}
\begin{center}
{
\footnotesize
\resizebox{\textwidth}{!}{
\begin{tabular}{
|p{0.8\linewidth}|p{0.2\linewidth}|
}
\hline
\cellcolor{gray!15}
\textbf{Busca} &
\cellcolor{gray!15}
\textbf{Quantidade} \\
\hline
{\tiny ([[Abstract: "microservice"] OR [Abstract: "state machine replication"]] AND NOT [[Abstract: "blockchain"] OR [Abstract: "network security"]] AND [[Title: "microservice"] OR [[Title: "state machine replication"] AND NOT [[Title: "blockchain"] OR [Title: "network security"]]]])} &
86 \\
\hline
\end{tabular}
}
}
\end{center}
\label{tab:extracao-ACM}
\end{table}

Dos trabalhos encontrados foram lidos os títulos e resumos, aplicando-se os critérios de inclusão e exclusão e critérios de qualidade. Foram encontrados 226 trabalhos no total. Desse total X artigos foram descartados por não atenderem os critérios.

\section{Aplicação dos Critérios e Listagem dos Trabalhos}

Foram listados os trabalhos resultantes das pesquisas nas bases científicas, bem como também Google Scholar. Essa listagem, em forma de tabela, está localizada no Anexo \ref{anexo:revisao}. O Anexo \ref{anexo:revisao} contém uma descrição breve dos artigos acolhidos pelas pesquisas bibliográficas. Assim é possível mapear uma relação dos mesmos, ao nosso trabalho. E ainda, a classificação é baseada nos critérios de qualidade, levando em consideração a tabela de dados relevantes, para serem encontrados dentro dos artigos.

\section{Considerações Finais}

Por meio desta revisão simplificada, do estado da arte, foi possível identificar trabalhos interessantes e relevantes para nossos critérios.

% ajusta o espaçamento dos parágrafos do resumo
\setlength{\absparsep}{18pt}
\begin{resumo}
\SingleSpacing

Recentemente, arquiteturas baseadas em microsserviços ganharam popularidade, em parte por causa do modelo de programação modular, acoplamento mínimo entre as partes e o suporte de plataformas de orquestração de contêineres. Ordenação de mensagens é uma estratégia que garante que todas as réplicas evoluam igualmente, aumentando-se os níveis de disponibilidade de serviços. Visando aplicações que usufruem de interfaces \gls{HTTP} para operar, este trabalho propõe uma implementação de interface de comunicação, sobre o protocolo HTTP e para um ordenador de mensagens. Sabe-se que orquestradores de contêineres oferecem replicação de forma automática, porém o serviço oferecido por orquestradores garante replicação de aplicações \textit{stateless}. O objetivo é continuar desenvolvendo um ordenador de mensagens transparente ao usuário, para isto este trabalho estende uma pesquisa iniciada pelo grupo, que propõe o Hermes, um interceptador de mensagens como serviço que usufrui de mecanismos de orquestração de contêineres para prover replicação e tolerância a falhas. O desenvolvimento do serviço de ordenação de mensagens contou com a implementação da interface que promove a comunicação no Hermes. A implementação possibilita que o Hermes possa tratar mensagens \gls{HTTP}. Ao final houve investigação de desempenho da implementação em casos específicos de vazão e latência. Os experimentos incluíram duas aplicações para avaliação de desempenho: uma aplicação de \textit{log} que recebe requisições HTTP e salva, em arquivo de disco e uma aplicação geradora de carga que envia requisições HTTP, podendo ser configurada por parâmetros. A investigação demonstrou que as latências capturadas nos geradores de carga apresentaram valores maiores para o sistema replicado quando comparado com o caso não-replicado, isto era esperado. O cenário de carga de 100\% POST, os experimentos se mostraram mais promissores. O caso onde existe 100\% de cargas GET os experimentos se mostraram melhores que no caso híbrido de 50\% GET e 50\% POST, por causa que existe 50\% de chance de múltiplos processos inserirem mais linhas no arquivo de \textit{log}. Finalmente, as comparações entre os cenários replicados e não-replicado mostraram que o ordenador de mensages prove tolerância a falhas e replicação ativa de aplicações \textit{stateful} baseadas em HTTP.

\textbf{Palavras-chave}: Protocolo de consenso. Proxy. Interceptação de Mensagens. Orquestração de contêineres. Ordenação total. Arquitetura de microsserviços. Kubernetes. Docker.
\end{resumo}
